\documentclass{article}
\usepackage[paperwidth=9cm,paperheight=9cm, margin=0.5cm]{geometry}
\usepackage{enumitem}
\pagenumbering{gobble}
\usepackage[ngerman,british]{babel}
\usepackage[utf8]{inputenc}
\usepackage[T1]{fontenc}
\usepackage{microtype}
\usepackage{lmodern}
\usepackage{amsthm}
\usepackage{amsmath}
\usepackage{amsfonts}
\usepackage{amssymb}
\usepackage{mathtools}
\usepackage{wasysym}
\usepackage{faktor}
\usepackage{tikz}


\begin{document}
\setlength{\parindent}{0pt}
%Problem hier
%Bei a), b), ... bitte dieses benutzen
%\begin{enumerate}[label=\alph*)]
%\end{enumerate}
Consider a polynomial$$f(x)=x^{2012}+a_{2011}x^{2011}+\dots+a_{1}x+a_{0}.$$Albert Einstein and Homer Simpson are playing the following game. In turn, they choose one of the coefficients $a_{0},\dots,a_{2011}$ and assign a real value to it. Albert has the first move. Once a value is assigned to a coefficient, it cannot be changed any more. The game ends after all the coefficients have been assigned values.\newline
Homer's goal is to make $f(x)$ divisible by a fixed polynomial $m(x)$ and Albert's goal is to prevent this.
\begin{enumerate}[label=\alph*)]
\item Which of the players has a winning strategy if $m(x)=x-2012$?
\item Which of the players has a winning strategy if $m(x)=x^{2}+1$?
\end{enumerate}
\end{document}
