\documentclass{article}
\usepackage[paperwidth=9cm,paperheight=9cm, margin=0.5cm]{geometry}
\usepackage{enumitem}
\pagenumbering{gobble}
\usepackage[ngerman,british]{babel}
\usepackage[utf8]{inputenc}
\usepackage[T1]{fontenc}
\usepackage{microtype}
\usepackage{lmodern}
\usepackage{amsthm}
\usepackage{amsmath}
\usepackage{amsfonts}
\usepackage{amssymb}
\usepackage{mathtools}
\usepackage{wasysym}
\usepackage{faktor}
\usepackage{tikz}


\begin{document}
\setlength{\parindent}{0pt}
%Problem hier
%Bei a), b), ... bitte dieses benutzen
%\begin{enumerate}[label=\alph*)]
%\end{enumerate}
Fix positive integers \(\displaystyle n\) and \(\displaystyle k\) such that \(\displaystyle 2\leq k\leq n\) and a set \(\displaystyle M\) consisting of \(\displaystyle n\) fruits. A \emph{permutation} is a sequence \(\displaystyle x=(x_1,x_2,\ldots,x_n)\) such that \(\displaystyle \{x_1,\ldots,x_n\}=M\). Ivan \emph{prefers} some (at least one) of these permutations. He realized that for every preferred permutation \(\displaystyle x\), there exist \(\displaystyle k\) indices \(\displaystyle i_1<i_2<\ldots<i_k\) with the following property: for every \(\displaystyle1\le j<k\), if he swaps \(\displaystyle x_{i_j}\) and \(\displaystyle x_{i_{j+1}}\), he obtains another preferred permutation.\newline Prove that he prefers at least \(\displaystyle k!\) permutations.
\end{document}
