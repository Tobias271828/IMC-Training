\documentclass{article}
\usepackage[paperwidth=9cm,paperheight=9cm, margin=0.5cm]{geometry}
\usepackage{enumitem}
\pagenumbering{gobble}
\usepackage[ngerman,british]{babel}
\usepackage[utf8]{inputenc}
\usepackage[T1]{fontenc}
\usepackage{microtype}
\usepackage{lmodern}
\usepackage{amsthm}
\usepackage{amsmath}
\usepackage{amsfonts}
\usepackage{amssymb}
\usepackage{mathtools}
\usepackage{wasysym}
\usepackage{faktor}
\usepackage{tikz}


\begin{document}
\setlength{\parindent}{0pt}
%Problem hier
%Bei a), b), ... bitte dieses benutzen
%\begin{enumerate}[label=\alph*)]
%\end{enumerate}
Let \(\displaystyle p\) be a prime number and let \(\displaystyle k\) be a positive integer. Suppose that the numbers \(\displaystyle a_i=i^k+i\) for \(\displaystyle i=0,1,\ldots,p-1\) form a complete residue system modulo \(\displaystyle p\). What is the set of possible remainders of \(\displaystyle a_2\) upon division by \(\displaystyle p\)?
\end{document}
