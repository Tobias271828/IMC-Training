\documentclass{article}
\usepackage[paperwidth=9cm,paperheight=9cm, margin=0.5cm]{geometry}
\usepackage{enumitem}
\pagenumbering{gobble}
\usepackage[ngerman,british]{babel}
\usepackage[utf8]{inputenc}
\usepackage[T1]{fontenc}
\usepackage{microtype}
\usepackage{lmodern}
\usepackage{amsthm}
\usepackage{amsmath}
\usepackage{amsfonts}
\usepackage{amssymb}
\usepackage{mathtools}
\usepackage{wasysym}
\usepackage{faktor}
\usepackage{tikz}


\begin{document}
\setlength{\parindent}{0pt}
%Problem hier
%Bei a), b), ... bitte dieses benutzen
%\begin{enumerate}[label=\alph*)]
%\end{enumerate}
Ivan writes the matrix \(\displaystyle \begin{pmatrix}
	2 & 3 \\
	2 & 4
\end{pmatrix}\) on the board. Then he performs the following operation on the matrix several times: 
\begin{itemize}
\item he chooses a row or a column of the matrix, and
\item he multiplies or divides the chosen row or column entry-wise by the other row or column, respectively.
\end{itemize}
 Can Ivan end up with the matrix \(\displaystyle \begin{pmatrix}
	2 & 4 \\
	2 & 3
\end{pmatrix}\) after finitely many steps?
\end{document}
