\documentclass{article}
\usepackage[paperwidth=9cm,paperheight=9cm, margin=0.5cm]{geometry}
\usepackage{enumitem}
\pagenumbering{gobble}
\usepackage[ngerman,british]{babel}
\usepackage[utf8]{inputenc}
\usepackage[T1]{fontenc}
\usepackage{microtype}
\usepackage{lmodern}
\usepackage{amsthm}
\usepackage{amsmath}
\usepackage{amsfonts}
\usepackage{amssymb}
\usepackage{mathtools}
\usepackage{wasysym}
\usepackage{faktor}
\usepackage{tikz}


\begin{document}
\setlength{\parindent}{0pt}
%Problem hier
%Bei a), b), ... bitte dieses benutzen
%\begin{enumerate}[label=\alph*)]
%\end{enumerate}
For $n\ge1$ let $M$ be an $n\times n$ complex matrix with distinct eigenvalues $\lambda_{1},\lambda_{2},\dots,\lambda_{k}$ with multiplicities $m_{1},m_{2},\dots,m_{k}$, respectively. Consider the linear operator $L_{M}$ defined by $L_{M}(X)=MX+XM^{T}$, for any complex $n\times n$ matrix $X$. Find its eigenvalues and their multiplicities. ($M^{T}$ denotes the transpose of $M$; that is, if $M=(m_{k,l})$, then $M^{T}=(m_{l,k})$.)
\end{document}
