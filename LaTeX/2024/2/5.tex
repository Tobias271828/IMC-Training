\documentclass{article}
\usepackage[paperwidth=9cm,paperheight=9cm, margin=0.5cm]{geometry}
\usepackage{enumitem}
\pagenumbering{gobble}
\usepackage[ngerman,british]{babel}
\usepackage[utf8]{inputenc}
\usepackage[T1]{fontenc}
\usepackage{microtype}
\usepackage{lmodern}
\usepackage{amsthm}
\usepackage{amsmath}
\usepackage{amsfonts}
\usepackage{amssymb}
\usepackage{mathtools}
\usepackage{wasysym}
\usepackage{faktor}
\usepackage{tikz}


\begin{document}
\setlength{\parindent}{0pt}
%Problem hier
%Bei a), b), ... bitte dieses benutzen
%\begin{enumerate}[label=\alph*)]
%\end{enumerate}
We say that a square-free positive integer $n$ is \emph{almost prime} if$$ n \mid x^{d_1}+x^{d_2}+...+x^{d_k}-kx $$for all integers $x$, where $1=d_1<d_2<\dots<d_k=n$ are all the positive divisors of $n$. Suppose that $r$ is a Fermat prime (i.e., it is a prime of the form $2^{2^m}+1$ for an integer $m\geq 0$), $p$ is a prime divisor of an almost prime integer $n$, and $p\equiv 1\pmod r$. Show that, with the above notation, $d_i\equiv 1\ (\mathrm{mod}\ r)$ for all $1\leq i\leq k$.\newline
(An integer $n$ is called \emph{square-free} if it is not divisible by $d^2$ for any integer $d>1$.)
\end{document}
