\documentclass{article}
\usepackage[paperwidth=9cm,paperheight=9cm, margin=0.5cm]{geometry}
\usepackage{enumitem}
\pagenumbering{gobble}
\usepackage[ngerman,british]{babel}
\usepackage[utf8]{inputenc}
\usepackage[T1]{fontenc}
\usepackage{microtype}
\usepackage{lmodern}
\usepackage{amsthm}
\usepackage{amsmath}
\usepackage{amsfonts}
\usepackage{amssymb}
\usepackage{mathtools}
\usepackage{wasysym}
\usepackage{faktor}
\usepackage{tikz}


\begin{document}
\setlength{\parindent}{0pt}
%Problem hier
%Bei a), b), ... bitte dieses benutzen
%\begin{enumerate}[label=\alph*)]
%\end{enumerate}
Let \(\displaystyle \Omega=\{(x,y,z)\in \mathbb{Z}^3: y+1\ge x\ge y\ge z\ge 0\}\). A frog moves along the points of \(\displaystyle \Omega\) by jumps of length $1$. For every positive integer \(\displaystyle n\), determine the number of paths the frog can take to reach \(\displaystyle (n,n,n)\) starting from \(\displaystyle (0,0,0)\) in exactly \(\displaystyle 3n\) jumps.
\end{document}
