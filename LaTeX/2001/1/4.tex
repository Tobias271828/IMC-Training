\documentclass{article}
\usepackage[paperwidth=9cm,paperheight=9cm, margin=0.5cm]{geometry}
\usepackage{enumitem}
\pagenumbering{gobble}
\usepackage[ngerman,british]{babel}
\usepackage[utf8]{inputenc}
\usepackage[T1]{fontenc}
\usepackage{microtype}
\usepackage{lmodern}
\usepackage{amsthm}
\usepackage{amsmath}
\usepackage{amsfonts}
\usepackage{amssymb}
\usepackage{mathtools}
\usepackage{wasysym}
\usepackage{faktor}
\usepackage{tikz}


\begin{document}
\setlength{\parindent}{0pt}
%Problem hier
%Bei a), b), ... bitte dieses benutzen
%\begin{enumerate}[label=\alph*)]
%\end{enumerate}
Let $k$ be a positive integer. Let $p(x)$ be a polynomial of degree $n$ each of whose coefficients is $-1$, $1$ or $0$, and which is divisible by $(x-1)^{k}$. Let $q$ be a prime such that $\frac{q}{\ln q}<\frac{k}{\ln(n+1)}$. Prove that the complex $q$th roots of unity are roots of the polynomial $p(x)$.
\end{document}
