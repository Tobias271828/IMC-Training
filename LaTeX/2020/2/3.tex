\documentclass{article}
\usepackage[paperwidth=9cm,paperheight=9cm, margin=0.5cm]{geometry}
\usepackage{enumitem}
\pagenumbering{gobble}
\usepackage[ngerman,british]{babel}
\usepackage[utf8]{inputenc}
\usepackage[T1]{fontenc}
\usepackage{microtype}
\usepackage{lmodern}
\usepackage{amsthm}
\usepackage{amsmath}
\usepackage{amsfonts}
\usepackage{amssymb}
\usepackage{mathtools}
\usepackage{wasysym}
\usepackage{faktor}
\usepackage{tikz}


\begin{document}
\setlength{\parindent}{0pt}
%Problem hier
%Bei a), b), ... bitte dieses benutzen
%\begin{enumerate}[label=\alph*)]
%\end{enumerate}
Let \(\displaystyle G\) be a group and \(\displaystyle n\ge2\) be an integer. Let \(\displaystyle H_1\) and \(\displaystyle H_2\) be two subgroups of \(\displaystyle G\) that satisfy$$[G:H_1]=[G:H_2]=n$$and$$[G:(H_1\cap H_2)]=n(n-1).$$Prove that \(\displaystyle H_1\) and \(\displaystyle H_2\) are conjugate in \(\displaystyle G\).\newline (Here \(\displaystyle [G:H]\) denotes the \emph{index} of the subgroup \(\displaystyle H\), i.e. the number of distinct left cosets \(\displaystyle xH\) of \(\displaystyle H\) in \(\displaystyle G\). The subgroups \(\displaystyle H_1\) and \(\displaystyle H_2\) are \emph{conjugate} if there exists an element \(\displaystyle g\in G\) such that \(\displaystyle g^{-1}H_1g=H_2\).) 
\end{document}
